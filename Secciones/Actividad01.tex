\section{Introdución} 
\vspace{12mm} %5mm vertical space


Con la promulgación de la Ley No 30220, Ley Universitaria, el Ministerio de Educación (MINEDU) asume la rectoría de la Política de Aseguramiento de la Calidad de la Educación Superior Universitaria. Además, se crea la Superintendencia Nacional de Educación Superior Universitaria (SUNEDU), y se introduce el licenciamiento obligatorio y renovable de las universidades, en lugar de la autorización de funcionamiento provisional y definitiva del anterior marco legal.
El diseño del modelo de licenciamiento se enmarca en la Política de Aseguramiento de la Calidad de la
Educación Superior Universitaria. En ella, el licenciamiento, conjuntamente con la acreditación, el fomento y los sistemas de información, conforman los cuatro pilares del Sistema de Aseguramiento de la Calidad (SAC). En dicho sistema, el licenciamiento opera como un mecanismo de protección del bienestar individual y social al no permitir la existencia de un servicio por debajo de las condiciones básicas de calidad (en adelante, CBC).
En el marco del SAC, la acreditación y el licenciamiento se definen como procesos distintos, pero a su vez complementarios, de evaluación de la calidad. Mientras que la acreditación es voluntaria, el licenciamiento es un requisito obligatorio para el funcionamiento de las universidades. Además, las CBC del licenciamiento constituyen un primer nivel para ofrecer un servicio de calidad, mientras que la acreditación se encuentra en un nivel superior, puesto que supera las condiciones mínimas de calidad y posee una dinámica orientada hacia la excelencia académica.


		

